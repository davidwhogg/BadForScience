% To do:
% ------
% - Write a title and abstract.
% - Write an outline.
% - Fill in the outline.
% - Publish.
% - Wait for the call from Stockholm.

% Style notes:
% ------------
% - Uhhh.

\documentclass[modern]{aastex631}
\usepackage[utf8]{inputenc}

% Hogg typesetting issues
\addtolength{\textheight}{0.8in}
\addtolength{\topmargin}{-0.4in}
\setlength{\parindent}{1.2\baselineskip} % seriously
\frenchspacing\raggedbottom\sloppy\sloppypar

% Math issues
\newcommand{\teff}{T_{\mathrm{eff}}}
\newcommand{\logg}{\log g}
\newcommand{\feh}{[\mathrm{Fe}/\mathrm{H}]}

\shorttitle{machine learning and science}
\shortauthors{hogg}

\begin{document}

\title{Is machine learning good or bad for science?}

\author{David W. Hogg}

\begin{abstract}\noindent % seriously!
  Machine-learning methods are having a huge impact across the natural sciences.
  Advances have come from classification, regression, dimensionality reduction, clustering, and many other things.
  However, machine learning has a strong ontology---for machine-learning methods, only the data exist---and a strong epistemology---models are generally considered good if they perform well on held-out training data---and these philosophies are in strong conflict from standard practice and philosophy in the natural sciences.
  Here I identify uses for machine learning in the natural sciences where the ontology and epistemology are valuable, and will help scientific projects.
  I also call out uses for machine learning that are not a good idea.
  In particular, there are a lot of reasons to be concerned about the use of complex machine-learning models to emulate physical (or first-principles) simulations.
  Finally, I conclude by noting that there are places in scientific investigations in which the choice to use a machine-learning method is the most conservative possible choice.
  These places are generally in projects that look like causal inferences, where the machine-learning method is used to model the possible effects of confounders.
  I illustrate the above points with examples from astrophysics.
\end{abstract}

\section{Introduction}\label{sec:intro}

Hello World!

\section{What is machine learning?}\label{sec:what}

\section{Stuff in between}\label{sec:stuff}

\section{Discussion}\label{sec:discussion}

Hello World!

\bibliography{sample631}{}
\bibliographystyle{aasjournal}

\end{document}
